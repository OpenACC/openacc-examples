It is important to recognize the difference between independent and dependent data. Independent data is data whose value does not depend on other data. Meanwhile, dependent data is data where the value depends on the value of another piece of data. If a piece of data can be taken on it's own without relying on other data, it is independent; otherwise, it is dependent. 

The example below details how data is implicitly declared as independent. Data is implicitly declared independent when there is no auto or seq clause in the loop construct, or when it is a orphaned loop construct or the parent compute region is a parallel construct. 

\begin{Code}
\begin{lstlisting}[frame=single, caption=dataIndependence.c, label=prototype, numbers=none]
double * Data1 = (double *)malloc( 100 * sizeof(double));
double * Data2 = (double *)malloc( 100 * sizeof(double));

for( int x = 0; x < 100; ++x){
    Data1[x] = x;
    Data2[x] = x;
}

int error = 0;
#pragma acc parallel loop copy(Data1[0:100])
for(int x = 0; x < 100; ++x){
    Data1[x] += Data1[x];
}

#pragma acc parallel copy(Data2[0:100])
{
    #pragma loop
    for(int x = 0; x < 100; ++x){
        Data2[x] += Data2[x];
    }
}
for( int x = 0; x < 100; ++x){
    if(Data1[x] - Data2[x] > .00001){
        error += 1;
    }
}
\end{lstlisting}
\end{Code}