This is a collection of coding examples that is meant to supplement The OpenACC Application Programming Interface for version 3.2. It is not meant to replace the specification, and is also not a part of the formal specification itself. This guide also assumes familiarity with OpenACC specifications. 

The OpenACC MPI specification provides a model for parallel programming that is portable across operating systems and various types of multicore CPUs and accelerators. The directives extend the ISO/ANSI standard C, C++, and Fortran base languages in a way that allows a programmer to migrate applications incrementally to parallel multicore and accelerator targets using standards-based C, C++, or Fortran. The programming model and OpenACC API compilers and run-time environment handles managing data, transfer between a host and accelerator, and initiate accelerator startup and shutdown. The model also allows augmenting information available to the compilers, including specification of data local to an accelerator, mapping of loops for parallel execution, and more.

Additional examples are in development and will be released in future iterations of this guide. Complete information about the OpenACC API and compilers that support the OpenACC API can be found at the OpenACC website\footnote{OpenACC Website: https://www.openacc.org/sites/default/files/inline-images/Specification/OpenACC-3.2-final.pdf}.

This document is being assembled by the OpenACC Verification and Validation Testsuite. If you would like more information on the Testsuite, please visit their official website\footnote{OpenACC V-V Testsuite: https://www.openacc.org/sites/default/files/inline-images/Specification/OpenACC-3.2-final.pdf}.

Unless stated otherwise, codes in this document were created by the OpenACC V-V team. Any codes contributed to this document outside of the OpenACC V-V team are credited alongside their code. 

All examples in this document are contained in the OpenACC Examples GitHub repository\footnote{OpenACC Examples repository: https://github.com/OpenACC/openacc-examples}. This repository will be updated alongside this document. 