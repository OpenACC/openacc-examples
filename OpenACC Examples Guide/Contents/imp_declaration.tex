A simple yet vital concept to understand is implicit versus explicit declaration. Explicit variable declaration is simply when a variable's type is defined before the value is set. So, a variable x is declared as an integer, and then later in the code x is set to some integer value. Implicit declaration means a variable's type is not defined ahead of time, and is stead assumed by the operators; however, any data could still be put into it. An example of this in action is shown below; notice how the variable sum is implicitly declared by just being set to the values of \texttt{device\_values}[n]. 

\begin{Code}
\begin{lstlisting}[frame=single, caption=combinedConstructs.c, label=prototype, numbers=none]
int sum = 0;
double *device_values = (double *)malloc(100 * sizeof(double) );

for(int x = 0; x < 100; ++x){
    device_values[0] = 1;
}

#pragma acc parallel loop reduction(+:sum)
for( int x = 0; x < 100; ++x){
    sum += device_values[0];
}
\end{lstlisting}
\end{Code}

Please note that aggregate variables are implicitly declared within a copy clause if the variable is not within a data clause already and the default none is not used within a the pragma. 