The enter data directive defines variables to be allocated in current device memory for the full data lifetime, or until an exit data directive deallocates the data. The enter data directive determines whether data should be copied from local to device memory. The exit data directive determines if data should be copied from device to local memory\cite{OpenACCSpec}.

In the code below, the use of the enter data creates a data region with copyin making copes of the data. After the first pragma, we enter a compute region wrapped in a data region that must declare the variables to use. Due to the fact that the structured reference counter has been incremented by the enter data clause, the dynamic reference counter has been incremented by the present clause, and this copyout only decrements the dynamic reference counter, the copyput action will only be performed when the exit data decrements the structured reference counter. Once the structured and dynamic reference counters are 0, the copyout action is then performed.

\begin{Code}
\begin{lstlisting}[frame=single, caption=enterDataExitData.c, label=prototype, numbers=none]
double *Data1 = (double *)malloc( 100 * sizeof(double));
double *Data2 = (double *)malloc( 100 * sizeof(double));
double *Data3 = (double *)malloc( 100 * sizeof(double));
double sum1 = 0.0;
double sum2 = 0.0;
double sum3 = 0.0;
double test1 = 0.0;
double test2 = 0.0;
double test3 = 0.0;
int error = 0;
for( int x = 0; x < 100; ++x){
    Data1[x] = x;
    Data2[x] = 2 * x;
    Data3[x] = 3 * x;
    test1 += Data1[x];
    test2 += Data2[x];
    test3 += Data3[x];
}
\end{lstlisting}
\end{Code}

\begin{Code}
\begin{lstlisting}[frame=single, caption=enterDataExitData.c (cont.), label=prototype, numbers=none]
#pragma acc enter data copyin(Data1[0:100], Data2[0:100], Data3[0:100], sum1, sum2, sum3)

#pragma acc parallel loop present(Data1[0:100], sum1) reduction(+:sum1) copyout(sum1)
for(int x = 0; x < 100; ++x){
    sum1 += Data1[x];
}

#pragma acc parallel loop present(Data2[0:100], sum2) reduction(+:sum2) copyout(sum2)
for(int x = 0; x < 100; ++x){
    sum2 = Data2[x];
}

#pragma acc parallel loop present(Data3[0:100], sum3) reduction(+:sum3) copyout(sum3)
for(int x = 0; x < 100; ++x){
    sum3 = Data3[x];
}
#pragma acc exit data copyout(sum1, sum2, sum3)

if( test1 != sum1){
    error += (1 << 0);
}
if( test2 != sum2){
    error += (2 << 0);
}
if( test3 != sum3){
    error += (3 << 0);
}
\end{lstlisting}
\end{Code}
